\documentclass[12pt]{article}
\usepackage[b]{mymacros}
\usepackage{hyperref}
\usepackage[dutchcal, slant]{suppose}

\begin{document}
    \tableofcontents
    \newpage

    \section{Schema}
    \subsection{Entity Sets}
    function(\ul{function-id}, oddCoefficient, oddAddend, evenDivisor) \\
    evaluation(\ul{eval-id}, \textit{function-id}, value) \\
    chain(\ul{chain-id}, \textit{eval-id}, chain) \\
    loop(\ul{loop-id}, \textit{chain-id}, loop, stabilization)

    \subsection{Relations}
    \begin{tabular}{l|l}
        evaluation(function, chain) & $[1..1] \to [0..n],[0..m]$ \\
    \end{tabular}

    \subsection{Tables}
    evaluation(function-id, chain-id, value)

    \newpage
    \section{Lemmas}
    \begin{gather*}
        Converge(x_n, L) :\equiv
        (\fall \ve > 0)(\ex N)[n > N \ra |x_n - L| < \ve] \\
        Diverge(x_n) :\equiv
        (\fall L)(\ex \ve > 0)(\fall N)[n > N \ra |x_n - L| \ge \ve] \\
    \end{gather*}

    \begin{lemma}
        $(2^k, 2^km + 2^k - \ell)$, $2 \le \ell \le 2^k - 2$
    \end{lemma}
    \begin{proof}
        $\supp x \equiv 1 \pmod 2$. \\
        $x, \\
        2^kx + 2^km + 2^k - \ell, \\
        2^{k - 1}x + 2^{k - 1}m + 2^{k - 1} - \f{\ell}{2}, \\
        2^k(2^{k - 1}x + 2^{k - 1}m + 2^{k - 1} - \f{\ell}{2}) + 2^km + 2^k - \ell \\
        2^{2k - 1}x + 2^{2k - 1}m + 2^{2k - 1} - \ell 2^{k - 1} + 2^km + 2^k - \ell$
    \end{proof}

    \begin{lemma}
        $(a, b, 2)$ goes to infinity if and only if $(b, a, 2)$ does
    \end{lemma}
    \begin{proof}
        $\supp (a, b, 2)$ goes to infinity. \\
        $\supp x \equiv 1 \pmod 2$ \\
        $\supp a,b$ even \\
        $x, ax + b, a'x + b', a(a'x + b') + b = aa'x + ab' + b, a'^2x + a'b' + b',\dots$ \\
        $x, bx + a, b'x + a', bb'x + ba' + a, b'^2x + b'a' + a'$ \\
        $\supp a,b$ odd \\
        $x, ax + b, (ax + b)', a(ax + b)' + b, (a(ax + b)' + b)', a(a(ax + b)' + b)' + b,\dots$ \\
        $x, bx + a, (bx + a)', b(bx + a)' + a, (b(bx + a)' + a)', b(b(bx + a)' + a)' + a,\dots$
    \end{proof}

    \begin{lemma}
    $(4, 4k + 2, 2)$, $k = 1,\dots$
    % 4 | 6, 10, 14, 18, 22, 26, 30
    \end{lemma}
    \begin{proof}
        $\supp x \equiv 1 \pmod 2$. The iterations of this function are 
        $x, 4x + 4k + 2, 
        2x + 2k + 1, 8x + 12k + 6, 
        4x + 6k + 3, 16x + 28k + 14, 
        8x + 14k + 7, 32x + 60k + 30, 
        16x + 30k + 15, 64x + 124k + 62, 
        32x + 62k + 31, 128x + 252k + 126, 
        64x + 126k + 63, 256x + 508k + 254, 
        128x + 254k + 127, 512x + 1020k + 510, 
        \dots$. We can rewrite these as $x, 
        2x + 2k + 1, 
        4x + 4k + 2 + 2k + 1, 
        8x + 8k + 4 + 4k + 2 + 2k + 1,\dots$, or even 
        $x, 
        2x + 2k + 1, 
        2(2x + 2k + 1) + 2k + 1, 
        4(2x + 2k + 1) + 2(2k + 1) + 2k + 1,\dots$. Hence, we can see that the sequence of odd numbers in this sequence is $\mseti{2^\ell(2x + 2k + 1) + \msum{m}{0}{\ell} m(2k + 1)}{\ell}{1}$, and the sequence of even numbers is double this. Since both sequences approach infinity as $\ell \to \infty$, this sequence of iterations goes to infinity. \p
        $\supp x \equiv 0 \pmod 2$. Then, $x$ reduces to an odd factor that initiates the above sequence. 
    \end{proof}


    \begin{lemma}
        $(4k + 2, 4, 2)$
    \end{lemma}
    \begin{proof}
        $\supp x \equiv 1 \pmod 2$ \\
        $x, 4xk + 2x + 4, \\
        2xk + x + 2, 8k^2x + (8x + 8)k + 2x + 8, \\
        4k^2x + (4x + 4)k + x + 4, 16k^3x + (24x + 16)k^2 + (12x + 24)k + 2x + 12, \\
        8k^3x + (12x + 8)k^2 + (6x + 12)k + x + 6, 32k^4x + (64x + 32)k^3  + (48x + 64)k^2 + (16x + 48)k + 2x + 16, \\
        16k^4x + (32x + 16)k^3 + (24x + 32)k^2 + (8x + 24)k + x + 8, 64k^5x + (160x + 64)k^4 + (160x + 160)k^3 + (80x + 160)k^2 + (20x + 80)k + 2x + 20, \\
        32*k^5*x+(80*x+32)*k^4+(80*x+80)*k^3+(40*x+80)*k^2+(10*x+40)*k+x+10, \dots$ \\

        $x, \\
        2kx + x + 2, \\
        4k^2x + (4x + 4)k + x + 4, \\
        8k^3x + (12x + 8)k^2 + (6x + 12)k + x + 6, \\
        16k^4x + (32x + 16)k^3 + (24x + 32)k^2 + (8x + 24)k + x + 8, \\
        32k^5x + (80x + 32)k^4 + (80x + 80)k^3 + (40x + 80)k^2 + (10x + 40)k + x + 10, \dots$ \\
        $x, \\
        2kx + x + 2, \\
        4k^2x + \tcr{(4x + 4)k} + x + 4, \\
        8k^3x + \tcb{(12x + 8)k^2} + \tcr{(4x + 4)k + (2x + 8)k} + x + 6, \\
        16k^4x + \tcg{(32x + 16)k^3} + \tcb{(12x + 8)k^2 + (12x + 24)k^2} + \tcr{(4x + 4)k + (2x + 8)k + (2x + 12)k} + x + 8, \\
        32k^5x + (80x + 32)k^4 + \tcg{(32x + 16)k^3 + (48x + 54)k^3} + \tcb{(12x + 8)k^2 + (12x + 24)k^2 + (16x + 48)k^2} + \\
        \tcr{(4x + 4)k + (2x + 8)k + (2x + 12)k + (2x + 16k)} + x + 10, \dots$ \\

        $(2k)^\ell x + (something) + x + 2\ell$
    \end{proof}

    \begin{lemma}
        $(2k + 1, 2(k - 1) + 1)$, $x = 2k + 1$, $k \pmod 2 = 0$???
    \end{lemma}
    \begin{proof}
        $2k + 1, (2k + 1)^2 + 2k - 1 = 4k$
    \end{proof}
    
    \begin{lemma}
        We can say nothing of $(a, a, 2)$. 
    \end{lemma}
    \begin{proof}        
        $\supp a \pmod 2 \equiv 1$, and $x \pmod 2 \equiv 1$. The chain of this function at $x$ is $x, ax + a$. Since $ax + a \pmod 2 \equiv 0$, we can say nothing more here, nor can we say anything when we start with an $x$ for which $x \mod 2 \equiv 0$, since it merely reduces to the odd case. \p
        $\supp a \pmod 2 \equiv 0$ and $x \pmod 2 \equiv 1$. The chain of this function at $x$ starts as $x, ax + a$. Since $a \pmod 2 \equiv 0$, this becomes $a'x + a'$, where $a'$ is $a$ stripped of its factors of 2. Then, $a'x + a' \pmod 2 \equiv 0$, and, again, we can do nothing more. \p
        $\supp x \pmod 2 \equiv 0$. Then, $x$ reduces to an odd factor that initiates the above process and of which we can say nothing. 
    \end{proof}

    \begin{lemma}
        $(o, e, 2) \to \infty \fa x$
    \end{lemma}
    \begin{proof}
        $\supp x \pmod 2 \equiv 1$. Then, $ox + e \pmod 2 \equiv 1$. We continue to feed these odd numbers back into the function, and the sequence goes to infinity. \p
        $\supp x \pmod 2 \equiv 0$. Then, $x$ reduces to an odd factor $x'$ that initiates the above sequence, taking it to infinity. 
    \end{proof}

    % (4, 6, 2) at 1,...,6
    % (6, 4, 2) at 1,...,6
    % 4 = e
    
    \begin{lemma} \label{lemma:e,eh,2}
        $(e, eh, 2)$, where $e$ is an even positive integer and $h \pmod 2 \equiv 0$, goes to infinity. 
    \end{lemma}
    \begin{proof}
        $\supp x \pmod 2 \equiv 1$. The chain of this function starts as $x, ex + eh, x + h$. If $h \pmod 2 \equiv 1$, then $x + h \pmod 2 \equiv 0$, and we can't continue the sequence. If $h \pmod 2 \equiv 0$, then $x + h$ is odd, and we can continue the sequences as $e(x + h) + eh, x + 2h, e(x + 2h) + eh,\dots$, which can be split into the two sequences $\mseti{x + kh}{k}{0}$ and $\mseti{e(x + kh) + eh}{k}{0}$, which both diverge. \p
        $\supp x \pmod 2 \equiv 0$. Then, $x$ reduces to an odd factor that initiates the above sequences. 
    \end{proof}
    \begin{example}
        $(2, 8, 2) \to \infty \forall x$ and $(2, 12, 2) \to \infty \forall x$
    \end{example}

    \begin{lemma} \label{lemma:eh,e,2}
        $(eh, e, 2)$, where $e$ is an even positive integer and $h \pmod 2 \equiv 0$, goes to infinity. 
    \end{lemma}
    \begin{proof}
        $\supp x \pmod 2 \equiv 1$. This function's chain at $x$  begins as $x, ehx + e, hx + 1$. If $h \pmod 2 \equiv 1$, then $hx + 1 \pmod 2 \equiv 0$, and we can't continue the sequence. If $h \pmod 2 \equiv 0$, then $hx + 1 \pmod 2 \equiv 1$, and the sequence continues as $eh^2x + eh + e, h^2x + h + 1, eh^3x + eh^2 + eh + e,\dots$, which can be split into the two sequences $\mseti{h^kx + \msum{\ell}{0}{k - 1}h^\ell}{k}{1}$ and 
        $\mseti{eh^kx + \msum{\ell}{0}{k - 1} eh^\ell}{k}{1}$, both of which go to infinity. \p
        $\supp x \pmod 2 \equiv 0$. Then, $x$ reduces to an odd factor that initiates the above sequences. \\

        Let $(x_n) = \mseti{h^kx + \msum{\ell}{0}{k - 1}h^\ell}{k}{1}$ and $\supp Converge(x_n, L)$. Then, for every $\ve > 0$, there is an $N \in \N$, such that $|x_n - L| < \ve$ when $n > N$. \\

        Let $L \in \R$ and let $\ve = \f{1}{2}\left(h^kx + \msum{\ell}{0}{k - 1}h^\ell\right)$. 
    \end{proof}
    \begin{example}
        $(12, 2, 2) \to \infty \forall x$ and $(12, 6, 2) \to \infty \forall x$
    \end{example}

    FIX: \dots We could just look at the result of stripping an even number of all its factors of 2. \\
    \begin{comment}
    \begin{lemma}
        $(4, 12k + 6, 2) \to \infty$
    \end{lemma}
    \begin{proof}
        $1, 10 + 12k, 5 + 6k, 26 + 36k, 13 + 18k, 58 + 84k, 29 + 42k, 122 + 180k, 61 + 90k, 250 + 372k, 125 + 186k, 256 + 384k, 128 + 192k,\dots$ \\
        $1, \\
        2(5 + 6k), 5 + 6k, \\
        2(13 + 18k), 13 + 18k, \\
        2(29 + 42k), 29 + 42k, \\
        2(61 + 90k), 61 + 90k, \\
        2(125 + 186k), 125 + 186k, \\
        2(128 + 192k), 128 + 192k,
        \dots$ \\

        $5 + 6k, 5 + 6k + (8 + 12k), 5 + 6k + (8 + 12k) + 2(8 + 12k),\dots$ \\
        $\mseti{2\bs(5 + 6k + \ell(8 + 12k)\bs)}{\ell}{0}$ and $\mseti{5 + 6k + \ell(8 + 12k)}{\ell}{0}$ \\

        $3, 18 + 12k, 9 + 6k, 42 + 36k, 21 + 18k, 90 + 84k, 45 + 42k, 186 + 180k, 93 + 90k, 
        378 + 372k, 189 + 186k, 
        762 + 756k, 381 + 378k, 
        1530 + 1524k, 765 + 762k, 
        3066 + 3060k, 1533 + 1530k,\dots \\
        3(1), \\
        6(3 + 2k), 3(3 + 2k), \\
        6(7 + 6k), 3(7 + 6k), \\
        6(15 + 14k), 3(15 + 14k), \\
        6(31 + 30k), 3(31 + 30k), \\
        6(63 + 62k), 3(63 + 62k), \\
        6(127 + 125k), 3(127 + 125k), \\
        6(255 + 253k), 3(255 + 253k), \\
        6(511 + 509k), 3(511 + 509k),
        \dots$ \\

        $3 + 2k, 3 + 2k + (4 + 4k), 3 + 2k + 2(4 + 4k),\dots$ \\
        $\mseti{6(3 + 2k + \ell(4 + 4k))}{\ell}{0}$ and $\mseti{3(3 + 2k + \ell(4 + 4k))}{\ell}{0}$ \\

        $\supp x \pmod 2 \equiv 1$ \\
        $x, \\
        4x + 12k + 6, 2x + 6k + 3, \\
        8x + 36k + 18, 4x + 18k + 9, \\
        16x + 84k + 42, 8x + 42k + 21, \\
        32x + 180k + 90, 16x + 90k + 45, \\
        64x + 372k + 186, 32x + 186k + 93, \\
        128x + 756k + 378, 64x + 378k + 189, \\
        256x + 1524k + 762, 128x + 762k + 381\dots$ \\

        $\mseti{2(2x + 6k + 3 + \ell(2x + 12k + 6))}{\ell}{0}$ and $\mseti{x + 6k + 3 + \ell(2x + 12k + 6)}{\ell}{0}$ \\

        $\supp x \pmod 2 \equiv 0$ \\
        $x, \dots$ ($x$ reduces to an odd number $x'$), $x'$, [same process as above]
    \end{proof}

    $(8, 2k, 2)$ \\
    $1, 8 + 2k, 4 + k,\dots$ Depends on $k$

    FIX: $(5, 2k + 1, 2) - (5, 1, 2), (5, 3, 2), (5, 5, 2),\dots, (5, 13, 2), (5, 15, 2), (5, 19, 2),\dots$ \\
    ($x$'s that go to $\infty$) \\
    $1: 7, 9, 11, 14, 18, 21, 22, 23, 25, 28, 29, 21, 35, 36, 37, 39, 41, 42, 44, 45, 46, 47, 49, 50,\dots \\ 
    3: 5, 7, 10, 11, 13, 14, 17, 19, 20, 21, 22, 23, 26, 27, 28, 29,\dots \\
    5: 13, 17, 21, 26, 27, 34, 35, 41, 42, 43, 45,\dots \\
    7: 19, 25, 29, 33, 38, 43, 45, 49, 50, 51, 58, 63, 65, 66, 67,\dots \\ 
    9: 15, 21, 30, 33, 39, 42, 51, 53, 57,\dots$ \\

    $1, 6 + 2k, 3 + k, 16 + 7k, \dots$ This depends on whether $k$ is odd or even. \\

    FIX: $(7, 2k + 1, 2) - (7, 11, 2), (7, 13, 2), (7, 15, 2), (7, 17, 2)$ \\
    $1, 8 + 2k, 4 + k,\dots$ Same thing \\
    \end{comment}

    \newpage
    \section{$x \bmod evenDivisor \neq 0$}
    $(3, 7, 3)$ \\
    $3, 16, 16 / 3$ \\

    $c > 1$ \\
    \[
        f(x) = 
    \begin{cases}
        e_1(x) & x \pmod c = c - 1 \\
        e_2(x) & x \pmod c = c - 2 \\
        \vdots \\
        e_{n-1}(x) & x \pmod c = 2 \\
        e_n(x) & x \pmod c = 1 \\
        x / c & x \pmod c = 0
    \end{cases}
    \]

    c = 2 \\
    \[
        f(x) = 
    \begin{cases}
        3n + 1 & x \pmod 2 \equiv 1 \\
        x / 2 & x \pmod 2 \equiv 0
    \end{cases} \\
    \]

    c = 2
    \[
        f(x) =
        \begin{cases}
            2n + 1 & x \pmod 2 \equiv 1 \\
            x / 2 & x \pmod 2 \equiv 0
        \end{cases}
    \]

    c = 3
    \[
        f(x) = 
        \begin{cases}
            3n + 2 & x \pmod 3 \equiv 2 \\
            3n + 1 & x \pmod 3 \equiv 1 \\
            x / 3 & x \pmod 3 \equiv 0
        \end{cases}
    \]
    or
    \[
        f(x) = 
        \begin{cases}
            3n + 1 & x \pmod 3 \equiv 2 \\
            3n + 2 & x \pmod 3 \equiv 1 \\
            x / 3 & x \pmod 3 \equiv 0
        \end{cases}
    \]
    This would have only one value $a = 3$. The others are determined from a. 
    \section{Entity Logic}
    \href{https://en.wikipedia.org/wiki/Second-order_logic}{STUFF TO READ}
    \subsection{Entity Set}
    Let $S$ be a set. $s \in S$ is an instance of $S$. There is a set of attributes $A(S)$ containing functions $f_i:E \to \R$ and relations $R_i:E \to \{True, False\}$, $i \in \N$. 

    e.g., $s \in S$,
    or: $John(s) = True, Smith(s) = True,\dots$ \\
    But then is $John$ the attribute, or $name$? \\

    What is $\N$ as an entity set or as a database? \\
    $A(\N) = \{prime, composite, even, odd, numPrimeFactors,f_1(x,t_2,\dots,t_n), \\
    f_2(x,t_2,\dots,t_n), \dots, y \mst R(x,y),\dots\}$ \\

    $A(\textsc{movement}) = \{velocity, momentum, friction, force, normal\_force,\dots\}$ \\

    What about constants of physics? What is there to be said \textit{about} them? \\
    $A(\{G, c, h, k, \dots\}) = \{\dots\}$ \\

    An entity set is what we say about a thing. A database is what we must say about at least two things. 

    \subsection{Relationship}
    \begin{definition}
        Let $E$ be an entity set, $e_1, e_2 \in E$, and $a \in A(E)$. We call $a$ the \textbf{primary key} of $E$ if $a(e_1) = a(e_2) \bic e_1 = e_2$. 
    \end{definition}

    \begin{theorem}
        Let $f:A \to B$ be a function. Then, $f$ is injective if and only if $f$ is a primary key for the entity set of $A$. 
    \end{theorem}
    \begin{proof}
        Let $f:A \inj B$ be an injective function. Then, $\fall x,y \in A$, $f(x) = f(y) \ra x = y$. Since $f$ is a function, we have that $x = y \ra f(x) = f(y)$, and, hence, $f$ is a primary key of $A$. \p
        Let $f \in A(A)$ be a primary key. So, $\fall x,y \in A, f(x) = f(y) \bic x = y$. Put $B = \R$. Then, $f:A \inj B$ is an injective function. 
    \end{proof}

    \begin{definition}
        Let $f:A \to B$ be a mapping. We call $f$ \textbf{ill-defined} if for some $x = y \in A$, $f(x) \neq f(y)$. 
    \end{definition}

    \begin{definition}
        Let $E_1$ and $E_2$ be entity sets, and let $P$ the primary key of $E_1$. We call $P$ a \textbf{foreign key} of $E_2$ if $Px \fa x \in E_1$ and $Py \fa y \in E_2$. 
    \end{definition}

    % relations:
    % 1) related by foreign keys
    % 2) related by tables

    \begin{definition}
        An ER-diagram is a graph with entity sets $E_1,\dots,E_n$ as vertices and with foreign keys $P_1,\dots,P_m$, $m \le \binom{n}{2}$ as edges. 
    \end{definition}

    \section{Case Theory}
    \begin{definition}
        Let $f:A \to B$ be a function and $c \in B$ a constant. We call a formula of the form $P(x) \ra f(x) = c$ a \textbf{case} of $f$. If $P$ is a modulus over some $n \in \N$, we call such a case a \textbf{modular case} or a \textbf{modular} $\mathbf{n}$\textbf{-case}. Moreover, we call a function defined only by modular cases a \textbf{modular function}. 
    \end{definition}

    \begin{definition}
        Let $f:A \to B$ be a function. If $f$ can be defined by an atomic formula, then, we call $f$ a \textbf{whole function}. If $f$ can defined by $n$ cases, then we call $f$ a \textbf{piecewise function} or an $\mathbf{n}$\textbf{-piece function}. 
    \end{definition}

    \begin{definition}
        Let $f:A \to B$ be an $n$-piece function defined by cases $C_1,\dots,C_n$. If one case $C_i$, $1 \le i \le n$, is a modular $m$-case defined as $x \equiv 0 \pmod m \ra f(x) = \f{x}{m}$, then we call $f$ a \textbf{stripping function}. 
    \end{definition}

    \begin{example}
        The function $f:\N \to \N$ given by 
        \[
            f(x) = 
            \begin{cases}
                3x + 1 & x \equiv 1 \pmod 2 \\
                \f{x}{2} & x \equiv 0 \pmod 2
            \end{cases}
        \]
        is a modular 2-piece stripping function defined by the cases $x \equiv 1 \pmod 2 \ra f(x) = 3x + 1$ and $x \equiv 0 \pmod 2 \ra f(x) = \f{x}{2}$. 
    \end{example}

    \begin{theorem}
        Let $f:A \to B$ be a whole function, where $A$ and $B$ are finite. Then, $f$ is an $n$-piece function.  
    \end{theorem}
    \begin{proof}
        Let $f:A \to B$, where $|A| = n \le m = |B|$, be a function given by the formula $f(x) = b$ for some $b \in B$. We have $A = \{a_1,\dots,a_n\}$ and $B = \{b_1,\dots,b_m\}$. Then, 
        \[
            f(x) = 
            \begin{cases}
                f(x) = b_1 & x = a_1 \\
                f(x) = b_2 & x = a_2 \\
                \vdots \\
                f(x) = b_n & x = a_n
            \end{cases}. 
        \]
        That is, we may define $f$ as $\Bigwedge_{i = 1}^n x = a_i \ra f(x) = b_i$. \p
        Now, let $f:A \to B$, where $|A| = n > m = |B|$, be a function given by the formula $f(x) = b$, where $b \in B$. We have $A = \{a_1,\dots,a_n\}$ and $B = \{b_1,\dots,b_m\}$. Then, 
        \[
            f(x) = 
            \begin{cases}
                f(x) = b_1 & x = a_1 \\
                f(x) = b_2 & x = a_2 \\
                \vdots \\
                f(x) = b_m & x = a_m \\
                f(x) = b_{i_1} & x = a_{m + 1} \\
                f(x) = b_{i_2} & x = a_{m + 2} \\
                \vdots \\
                f(x) = b_{i_{n - m}} & x = a_n
            \end{cases}, 
        \]
        where $1 \le i_j \le m$ for $j = 1,\dots,n - m$. That is, we may define $f$ as 
        \[
            \left(\Bigwedge_{i = 1}^m x = a_i \ra f(x) = b_i\right) \land 
            \left(
                \Bigwedge_{i = m + 1}^n
                \left(
                    \Bigwedge_{j = 1}^{n - m} x = a_i \ra f(x) = b_{i_j}
                \right)
            \right). 
        \]

        % \newpage
        % \noindent
        % $
        % x = a_{m + 1} \ra f(x) = b_{i_1} \land \\
        % x = a_{m + 1} \ra f(x) = b_{i_2} \land \\
        % \vdots \\
        % x = a_{m + 1} \ra f(x) = b_{i_{n - m}} \\
        % \bigwedge \\
        % x = a_{m + 2} \ra f(x) = b_{i_1} \land \\
        % x = a_{m + 2} \ra f(x) = b_{i_2} \land \\
        % \vdots \\
        % x = a_{m + 2} \ra f(x) = b_{i_{n - m}} \\
        % \bigwedge \\
        % \vdots \\
        % \bigwedge \\
        % x = a_n \ra f(x) = b_{i_1} \land \\
        % x = a_n \ra f(x) = b_{i_2} \land \\
        % \vdots \\
        % x = a_n \ra f(x) = b_{i_{n - m}}
        % $
    \end{proof}

    \begin{theorem}
        Let $f:A \to B$ be a whole function, where $|A| < \aleph_0 \le |B|$. Then, $f$ is an $n$-piece function. 
    \end{theorem}
    \begin{proof}
        Let $f$, $A$, and $B$ be as such. Then, $A = \{a_1,\dots,a_n\}$ and we may define $f$ with the formula $f(x) = b$ for some $b \in B$. Then, for some $b_1,\dots,b_n \in B$, 
        \[
            f(x) = 
            \begin{cases}
                f(x) = b_1 & x = a_1 \\
                f(x) = b_2 & x = a_2 \\
                \vdots \\
                f(x) = b_n & x = a_n
            \end{cases}, 
        \]
        and we may define $f$ as $\Bigwedge_{i = 1}^n x = a_i \ra f(x) = b_i$. 
    \end{proof}

    \begin{definition}
        Let $f:A \to B$ be a function. If $f$ is not piecewise, then a definition of $f$ in cases requires an infinite number of cases. We say that such a function is $\bs\infty$\textbf{-piece}. 
    \end{definition}

    \begin{theorem}
        Let $f:A \to B$ be a whole function, where $A$ is infinite. Then, $f$ is an $\infty$-piece function. 
    \end{theorem}
    \begin{proof}
        Let $f:A \to B$ be as such, and suppose $A$ is infinite. We have $A = \{a_0, a_1,\dots\}$. Suppose we define $f$ with the cases $f(a_0) = b_0$, $f(a_1) = b_1, \dots, f(a_n) = b_n$ for some $b_1,\dots,b_n \in B$. Then, there is an $a_{n + 1} \in A$ at which $f$ is not defined. Thus, we are no longer talking about $f$ but rather a function $g:S \to B$, where $S \subset A$ is finite. Therefore, $f$ is $\infty$-piece. 
    \end{proof}
    \begin{examples*}
        The successor function $S:\N \to \N$ given by $S(x) = x + 1$ is $\infty$-piece. \p
        $f:\N \to \R, f(x) = \msum{n}{1}{x} \f{1}{n}$ \p
        Any function whose domain contains $\N$. \p
        $\zeta:\{s \in \C \mid \Re(s) > 1\} \to \C$, $\zeta(s) = \msumi{n}{1} \f{1}{n^s}$
    \end{examples*}
    \section{Buchberger}
    $S_{ij} = \f{\lcm{g_i, g_j}}{g_i}f_i - \f{\lcm{g_i, g_j}}{g_j}f_j$ \\

    $f_i = p_1,\dots,p_n$, $f_j = p_1,\dots,p_m$ \\
    $p_i,p_j$ \\
    $S_{ij} = \f{\lcm{p_i, p_j}}{p_i}f_i - \f{\lcm{p_i, p_j}}{g_j}f_j$

    $6 = 2 \cdot 3$, $15 = 3 \cdot 5$ \\
    $S_{1,2} 
    = (2 \cdot 3 / 2)6 - (2 \cdot 3 / 3)15
    = 18 - 30 = -12$

    \[
        f\inv(x) = 
        \begin{cases}
            \{2x\} & x \equiv 0, 1, 2, 3, 5 \\
            \{2x, \f{x - 1}{3}\} & x \equiv 4
        \end{cases}
        \pmod 6
    \]
    $\supp x \equiv 4 \pmod 6$ \\
    $2x \equiv 2 \pmod 6$, $x \equiv 1 \pmod 6$ \\
    $x - 1 \equiv 3 \pmod 6, \f{x - 1}{3} \equiv 1 \mod 6$ \\

    $\supp x \equiv 0 \pmod 6$ \\
    $2x \equiv 0 \pmod 6, \f{x - 1}{3} \equiv 2 \pmod 6$ \\

    $\supp x \equiv 1 \pmod 6$ \\
    $2x \equiv 2 \pmod 6, \f{x - 1}{3} \equiv 0 \pmod 6$ \\

    $\supp x \equiv 2 \pmod 6$ \\
    $2x \equiv 4 \pmod 6, \f{x - 1}{3} \equiv 1 \pmod 6$ \\

    $\supp x \equiv 3 \pmod 6$ \\
    $2x \equiv 0 \pmod 6, \f{x - 1}{3} \equiv 2 \pmod 6$ \\

    $\supp x \equiv 5 \pmod 6$ \\
    $2x \equiv 4 \pmod 6, \f{x - 1}{3} \equiv 4 \pmod 6$
\end{document}